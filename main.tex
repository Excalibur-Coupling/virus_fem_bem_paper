% Modified for use with JCC - Madhusudan Singh Copyright (C) (2012). All rights reserved.
\documentclass[12pt]{article}

\setlength{\oddsidemargin}{0in}  %left margin position, reference is one inch
\setlength{\textwidth}{6.5in}    %width of text=8.5-1in-1in for margin
\setlength{\topmargin}{-0.5in}    %reference is at 1.5in, -.5in gives a start of about 1in from top
\setlength{\textheight}{9in}     %length of text=11in-1in-1in (top and bot. marg.) 
\newenvironment{wileykeywords}{\textsf{Keywords:}\hspace{\stretch{1}}}{\hspace{\stretch{1}}\rule{1ex}{1ex}}

\usepackage{amsmath,amssymb}
\usepackage{graphicx}% Include figure files
%\usepackage{caption}
\usepackage{color}% Include colors for document elements
\usepackage{dcolumn}% Align table columns on decimal point
\usepackage{bm}% bold math
\usepackage[numbers,super,comma,sort&compress]{natbib}
\usepackage{multirow}
%\usepackage[nolists, nomarkers, figuresfirst]{endfloat}


\usepackage{hyperref}

\definecolor{background-color}{gray}{0.98}

\title{Coupling finite and boundary element methods to solve the Poisson--Boltzmann equation for electrostatics in molecular solvation}
\author{Micha\l{} Bosy\thanks{School of Computer Science and Mathematics, Kingston University London, Penrhyn Road, Kingston upon Thames, KT1 2EE, UK}, 
    Matthew W. Scroggs\thanks{Department of Mathematics, University College London, 25 Gordon Street, WC1H 0AY London, UK}, 
    Timo Betcke\footnotemark[2],~\\%\thanks{Department of Mathematics, University College London, 25 Gordon Street, WC1H 0AY London, UK}, 
    Erik Burman\footnotemark[2],~%\thanks{Department of Mathematics, University College London, 25 Gordon Street, WC1H 0AY London, UK}, 
    Christopher D. Cooper\thanks{Department of Mechanical Engineering and Centro Cient\'ifico Tecnol\'ogico de Valpara\'iso, Universidad T\'ecnica Federico Santa Mar\'ia, Valpara\'iso 2390123, Chile}}

\begin{document}

\maketitle


\begin{abstract}
The Poisson--Boltzmann equation is widely used to model electrostatics in molecular systems. Available software packages solve it using finite difference, finite element, and boundary element methods, where the latter is attractive due to the accurate representation of the molecular surface and partial charges, and exact enforcement of the boundary conditions at infinity. However, the boundary element method is limited to linear equations and piecewise constant variations of the material properties. In this work, we present a scheme that couples finite and boundary elements for the Poisson--Boltzmann equation, where the finite element method is applied in a confined {\it solute} region, and the boundary element method in the external {\it solvent} region. As a proof-of-concept exercise, we use the simplest methods available: Johnson--N\'ed\'elec coupling with mass matrix and diagonal preconditioning, implemented using the Bempp-cl and FEniCSx libraries via their Python interfaces. We showcase our implementation by computing the polar component of the solvation free energy of a set of molecules using a constant and a Gaussian-varying permittivity. We validate our implementation against the finite difference code APBS (to 0.5\%), and show scaling from protein G B1 (955 atoms) up to immunoglobulin G (20\,148 atoms). For small problems, the coupled method was efficient, outperforming a purely boundary integral approach. For Gaussian-varying permittivities, which are beyond the applicability of boundary elements alone, we were able to run medium to large sized problems on a single workstation. Development of better preconditioning techniques and the use of distributed memory parallelism for larger systems remains an area for future work. We hope this work will serve as inspiration for future developments that consider space-varying field parameters, and mixed linear-nonlinear schemes for molecular electrostatics with implicit solvent models. 
\end{abstract}

\begin{wileykeywords}
Finite element method, Boundary element method, Poisson--Boltzmann, Implicit solvent model, Electrostatics.
\end{wileykeywords}

\clearpage

%*****************Graphical Table of Contents******************** THIS IS MANDATORY *******************


\begin{figure}[h]
\centering
\colorbox{background-color}{
\fbox{
\begin{minipage}{1.0\textwidth}
\includegraphics[width=50mm,height=50mm]{graphical_abstract_fem-bem.png} %Pick only one of the two styles by uncommenting the corresponding \includegraphics
%\includegraphics[width=110mm,height=20mm]{cc.eps}
\\
    The boundary element method is a popular numerical algorithm to solve the Poisson--Boltzmann equation for molecular electrostatics. However, this technique is limited to linear equations and piecewise constant variations of the field parameters. Here, we overcome such limitations by coupling it with a finite element method, implemented using the Bempp-cl and FEniCSx libraries via their Python interfaces. This results in an efficient, accurate, flexible, and easy-to-use computational tool for model development. 
\end{minipage}
}}
\end{figure}

% makes references listed with 1., 2., etc.  
0  \makeatletter
  \renewcommand\@biblabel[1]{#1.}
  \makeatother

\bibliographystyle{apsrev}

\renewcommand{\baselinestretch}{1.5}
\normalsize


\clearpage


\section*{\sffamily \Large INTRODUCTION} % Not needed for rapid communications

In biologically relevant settings, the structure and function of biomolecules are largely determined by the surrounding water, which usually contains salt. 
To describe these systems accurately, we need to account for the solvent correctly, which has given rise to a wide range of models.\cite{onufriev2018water}
Highly detailed models consider every water molecule and salt ion explicitly.
For solutions with large numbers of molecules, however, these models can be very computationally expensive, so implicit-solvent models---approximated models that use continuum theory to represent the ionic solution---are often used instead.\cite{RouxSimonson1999,DecherchiETal2015}
In the case of electrostatics, the implicit-solvent model is mathematically characterized by the Poisson--Boltzmann equation (PBE)\cite{Baker2004,Bardhan2012}, which is widely used to compute solvation free energies and mean-field potentials.

The implicit-solvent model for electrostatics describes the dissolved molecule as an infinite medium with a low-dielectric solute-shaped cavity, which contains a charge distribution from the partial charges---usually a sum of Dirac deltas at the atom's locations.
The outer solvent region is represented with a high dielectric constant and considers the presence of salt.
These two regions are interfaced by the molecular surface where the continuity of the electrostatic potential and electric displacement are enforced.
The molecular surface can be defined in various ways \cite{HarrisBoschitcshFenley2013}.


The PBE has been solved numerically with finite difference\cite{BakerETal2001,GilsonETal1985,JurrusETal2018,LiETal2019}, finite element\cite{HolstETal2012,BondEtal2010,nakov2021argos}, boundary element\cite{boschitsch2002fast,LuETal2006,AltmanBardhanWhiteTidor09,bajaj2011efficient,GengKrasny2013,CooperBardhanBarba2014}, and analytic\cite{YapHeadgordon2010,FelbergETal2017} methods.
In particular, the boundary element method (BEM) has proven to be very efficient for high-accuracy calculations \cite{GengKrasny2013,CooperBardhanBarba2014}, mainly due to the precise description of the molecular surface and point charges. 
However, BEM is limited to constant material properties in each region and the linear version of the PBE. 
Even though these limitations are acceptable in a wide range of applications, there are cases when BEM falls short: for example, if a variable permittivity is required inside the solute \cite{grant2001smooth,li2013dielectric}, or the solute is highly charged such that the linear approximation breaks\cite{FogolariETal1999}.

The present article describes a methodology to overcome some of those limitations, by coupling finite and boundary element methods.
This approach brings the best of both worlds---the flexibility of FEM and the efficiency of BEM---all in an accurate description of the solute molecule.
FEM-BEM coupling is a popular technique in the context of mixed linear-nonlinear models,\cite{carstensen1995coupling,aurada2013classical} fracture mechanics,\cite{aour2007coupled} fluid-structure interaction,\cite{estorff1991fem} acoustics,\cite{hiptmair2006stabilized} and electromagnetics.\cite{matsuoka1988calculation,hiptmair2008stabilized,bruckner20123d}
On the other hand, the PBE has been solved with hybrid numerical methods in the past: for example, by coupling finite differences with boundary elements\cite{boschitsch2004hybrid} or finite elements\cite{xie2016new,ying2018hybrid} to solve the nonlinear PBE in a specific region, and to implement modifications to the PBE model (i.e. the size-modified PBE).
To the best of our knowledge, this is the first time finite and boundary elements have been combined in this application.

In this paper, we prototype this principle with the simplest implementation possible: a Johnson--N\'ed\'elec\cite{johnson1980coupling} coupling, where we solve using mass-matrix and diagonal preconditioning and without distributed memory execution.
This limits the size of problems we can access currently, however, it sets the basis for future developments that use more elaborate formulations and algorithms that are readily available in open-source software libraries.
In this work, we use the boundary element library Bempp-cl\cite{BetckeScroggs2021} and the finite element library FEniCSx\cite{BasixJoss,BasixDofTransformations}.
These libraries are easy to use and their full functionalities may be accessed via their Python interfaces, making them ideal tools for easily implementing problems like this as well as for exploring computational efficiency and moving towards tackling large-scale problems, such as a full viral capsid.\cite{MartinezETal2019,wang2021high}
We hope this work will inspire research along these lines.


% The opening sentence of the manuscript should summarize the reasons for the undertaking of the work and the main conclusions that can be drawn.

%((Place Introduction here))

%((Main text paragraphs should be 12 point font, double-spaced. Reviews are comprehensive survey of recent progress in a topic of broad interest in quantum chemistry, providing the readership with an appreciation of the importance of the work, a summary of recent developments, and a guide to the relevant literature. Perspective are short discussions of an important emerging topics in quantum chemistry, usually focused on no more than a few recently published papers, and including the authors' vision for the future of the topics, identifying important problems that should be addressed next. Perspective should be limited to 3000 words, 4 display items (figures and/or tables), and 30 references.))

%((Full Papers are comprehensive reports of important recent advances in the development of basic theory, quantum mechanical computational methodologies and their relevant applications that provide significant insight to problems of broad interest in chemistry, physics, biology, and materials science. The opening sentence of the manuscript should summarize the reasons for the undertaking of the work and the main conclusions that can be drawn. The main text should be contain sections with brief subheadings, a summary of the major conclusions of the paper, and a Method section containing sufficient detail to reproduce the work. Main text paragraphs should be 12 point font, double-spaced.))

%((Rapid communications should be limited to 1500 words, 3 display items (figure and/or tables), 20 references. Main text paragraphs should be 12 point font, double-spaced. There are NO headings in the main text.))

\section*{\sffamily \Large METHODOLOGY}
\subsection*{\sffamily \large The implicit solvent model}

The implicit solvent model can be described mathematically as a coupled system of partial differential equations, where the Poisson-Boltzmann governs in the solvent region ($\Omega_1$ in Figure XX), and the Poisson equation in the solute region ($\Omega_2$ in Figure XX). These regions are interfaced by the molecular surface ($\Gamma$), where the potential ($\phi$) and electric displacement ($\epsilon\partial\phi/\partial\mathbf{n}$) are continuous. 
%
\begin{align}\label{eq:pbe}
\nabla^2\phi_1(\mathbf{x}) &= \frac{1}{\epsilon_1}\sum_{k=1}^{N_q} q_k\delta(\mathbf{x},\mathbf{x}_k) \quad  \mathbf{x} \in \Omega_1\nonumber\\
\left(\nabla^2 - \kappa^2\right)\phi_2(\mathbf{x})  &= 0 \quad\mathbf{x}\in\Omega_2\nonumber\\
\phi_1(\mathbf{x})  = \phi_2 (\mathbf{x}), &\quad \epsilon_1\frac{\partial\phi_1}{\partial\mathbf{n}}(\mathbf{x})  = \epsilon_2\frac{\partial\phi_2}{\partial\mathbf{n}}(\mathbf{x})  \quad \mathbf{x}\in \Gamma. 
\end{align}
%
where $\epsilon_1$ and $\epsilon_2$ are the dielectric constants in the solute and solvent, respectively, $\kappa$ is the inverse of the Debye length, related with the salt concentration, and $q_k$ are the values of the partial charges, located at $\mathbf{x}_k$.

The electrostatic potential in $\Omega_1$ can be further decomposed into singular and regular components as $\phi_1 = \phi_c + \phi_R$, where $\phi_c$ is the solution to
%
\begin{align}\label{eq:phic}
\nabla^2\phi_c(\mathbf{x}) &= \frac{1}{\epsilon}\sum_{k=1}^{N_q}q_k\delta(\mathbf{x},\mathbf{x}_k) \quad \mathbf{x}\in\Omega_1\cup\Omega_2\nonumber\\
\phi_c(\mathbf{x})&=0 \quad \text{ as } |\mathbf{x}|\to\infty
\end{align}

Physically, $\phi_c$ can be interpreted as the Coulomb-type potential from the point charges, whereas $\phi_R$, also known as reaction potential, is originated by the polarization of the solvent. 
Usually, $\epsilon_1$ is considered a constant value, yielding an analytical expression for $\phi_c$, however, this is not the general case.

There are regularized versions of Equation \eqref{eq:pbe} [HolstETal2008,GengZhao2020] which are widely used to numerically solve the Poisson-Boltzmann equation with finite element or finite difference methods.
However, here we use the standard formulation in Equation \eqref{eq:pbe}, as it offers more flexibility when dealing with, for example, variable permitivitties.

A common quantity of interest in implicit solvent models is the solvation free energy, which the change in Gibbs free energy as the molecule moves from vacuum into the solvent. Considering the charge distribution $\rho$ consists of point charges, this can be calculated as
%
\begin{equation}\label{eq:dG} 
\Delta G_{solv} = \frac{1}{2}\int_{\Omega_1} \rho(\mathbf{x})\phi_{r}(\mathbf{x}) = \frac{1}{2}\sum_{k=1}^{N_q} q_k\phi_r(\mathbf{x_k})
\end{equation}

\subsection*{\sffamily \large Numerical solution of the Poisson-Boltzmann equation}

\subsection*{\sffamily \large BEM-BEM coupling}

The boundary element method (BEM) is a standard tool for the numerical solution of the Poisson-Boltzmann equation in molecular electrostatics.~\cite{ZauharMorgan1985, Shaw1985} This was implemented in numerous codes, such as AFMPB,~\cite{LuETal2006} TABI,~\cite{GengKrasny2013} PyGBe,~\cite{CooperBardhanBarba2014,CooperForsythClementiBarba2016} and more recently, with Bempp-cl.~\cite{SearchCooperWout2022} 

There are several possible boundary integral formulations of Equation \eqref{eq:pbe}.~\cite{SearchCooperWout2022} The simplest form was presented by Yoon and Lenhoff in 1991,~\cite{YoonLenhoff1990} known as the \emph{direct} formulation, which only couples the potential at the interface, and not its derivative.
Applying Green's second identity to Equation \eqref{eq:pbe} we get 
%
\begin{align} \label{eq:volume_potential}
\phi_{1}+ K_{L}^{\Omega_1}(\phi_{1,\Gamma}) -  V_{L}^{\Omega_1} \left(\frac{\partial}{\partial \mathbf{n}}  \phi_{1,\Gamma}  \right) & = \frac{1}{\epsilon_1} \sum_{k=0}^{N_q}  \frac{q_k}{4\pi|\mathbf{r}_{\Omega_1} - \mathbf{r}_k|}  \quad \text{on $\Omega_1$,} \nonumber \\
\phi_{2} - K_{Y}^{\Omega_2}(\phi_{2,\Gamma}) + V_{Y}^{\Omega_2} \left( \frac{\partial}{\partial \mathbf{n}} \phi_{2,\Gamma} \right) & = 0 \quad \text{on $\Omega_2$,}
\end{align}
%
where $\phi_{1,\Gamma} = \phi_1(\mathbf{r}_\Gamma)$ and $\phi_{2,\Gamma} = \phi_2(\mathbf{r}_\Gamma)$ is the potential on $\Gamma$ as we approach from $\Omega_1$ and $\Omega_2$, respectively. $K$ and $V$ are the double- and single-layer potentials for the Laplace (subscript $L$) and Yukawa (subscript $Y$, also known as modified Helmholtz) kernels
%
\begin{align}\label{eq:single_double}
V^\Omega_{L,Y}(\varphi) = \oint_\Gamma g_{L,Y}(\mathbf{r}_\Omega,\mathbf{r}')\varphi(\mathbf{r}')d\mathbf{r}'\nonumber\\
K^\Omega_{L,Y}(\varphi) = \oint_\Gamma \frac{\partial g_{L,Y}}{\partial\mathbf{n}'}(\mathbf{r}_\Omega,\mathbf{r}')\varphi(\mathbf{r}')d\mathbf{r}',\nonumber\\
\end{align}
%
where $\varphi(\mathbf{r})$ can be any distribution over $\Gamma$, and 
%
\begin{align}\label{eq:green_func}
g_L(\mathbf{r},\mathbf{r}')=\frac{1}{4\pi|\mathbf{r}-\mathbf{r}'|} \nonumber \\
g_Y(\mathbf{r},\mathbf{r}')=\frac{e^{-\kappa|\mathbf{r}-\mathbf{r}'|}}{4\pi|\mathbf{r}-\mathbf{r}'|}
\end{align}
%
are the free-space Green's function of the Laplace and linearized Poisson-Boltzmann equations, respectively. 

In the \emph{direct} formulation, we take the limit of Equation \eqref{eq:volume_potential} as $\mathbf{r}$ approaches $\Gamma$and apply the interface conditions for $\phi$ and $\epsilon\partial\phi/\partial\mathbf{n}$ to get
%
\begin{align} \label{eq:direct}
\frac{\phi_{1,\Gamma}}{2}+ K_{L}^{\Gamma}(\phi_{1,\Gamma}) -  V_{L}^{\Gamma} \left(\frac{\partial}{\partial \mathbf{n}}  \phi_{1,\Gamma}  \right) & = \frac{1}{\epsilon_1} \sum_{k=0}^{N_q}  \frac{q_k}{4\pi|\mathbf{r}_{\Gamma} - \mathbf{r}_k|} \nonumber \\
\frac{\phi_{1,\Gamma}}{2} - K_{Y}^{\Gamma}(\phi_{1,\Gamma}) + \frac{\epsilon_1}{\epsilon_2}V_{Y}^{\Gamma} \left( \frac{\partial}{\partial \mathbf{n}} \phi_{1,\Gamma} \right) & = 0
\end{align}
%
Even though this formulation is poorly conditioned with respect to the mesh size, it can model large systems when it is preconditioned appropriately.\cite{AltmanBardhanWhiteTidor09,WangCooperBetckeBarba2021} 

\subsection*{\sffamily \large FEM-BEM coupling}

\begin{itemize}
    \item Description with constant and variable permittivity 
    
BEM reduces the dimension of the problem by using the boundary integral equation, hence its popularity as a numerical solution of  the Poisson-Boltzmann equation in molecular electrostatics. Unfortunately, such advantage comes with the cost. A fundamental solution must be found before the BEM can be applied. There are many linear problems for which fundamental solutions are not known. For Poisson-Boltzmann equation it is a case if we consider the heterogenous permittivity inside the molecule
   \begin{align} \label{eq:pbe_vp}
\epsilon_1(\mathbf{x}) \nabla^2\phi_1(\mathbf{x}) &= \sum_{k=1}^{N_q} q_k\delta(\mathbf{x},\mathbf{x}_k) \quad  \mathbf{x} \in \Omega_1\nonumber\\
\left(\nabla^2 - \kappa^2\right)\phi_2(\mathbf{x})  &= 0 \quad\mathbf{x}\in\Omega_2\nonumber\\
\phi_1(\mathbf{x})  = \phi_2(\mathbf{x}),  &\quad \epsilon_1(\mathbf{x})\frac{\partial\phi_1}{\partial\mathbf{n}}(\mathbf{x})  = \epsilon_2\frac{\partial\phi_2}{\partial\mathbf{n}}(\mathbf{x})  \quad \mathbf{x}\in \Gamma. 
\end{align}
For such case finite element methods (FEM) are more suitable.

The coupling of finite element (FEM) and boundary element (BEM) methods is the most widely used approach for solving multi-physical problems on an unbounded domain. It allows to take advantage of both methods. On the one hand, the BEM approximates only boundary conditions, hence it is commonly used for problems involving infinite or semi-infinite domains. On the other hand, the FEM is known for its robustness and universal applicability even for problem of inhomogeneous or non-linear nature.
And that is why, we present two formualtions of FEM-BEM coupling for Poisson-Boltzmann equation.
    
    \item Standard FEM-BEM coupling approach
   
   As mentioned before we will use finite element discretisation for internal problem on domain $\Omega_1$ and boundary elements for external domain $\Omega_2$. The formulation below can be applied for homogeneous and heterogeneous $\epsilon_1$, hence for simplicity we do not distinguish between these cases in the formulation.
    
    We start with the variational formulation of the internal problem. Applying integration by parts for first equation of~\eqref{eq:pbe_vp} for every $v \in H_0^1(\Omega_1)$ we have
\begin{equation}
\label{eq:fem}
 \int_{\Omega_1} \epsilon_1 \nabla \phi_1 : \nabla v ~d\mathbf{x}  - \oint_\Gamma \epsilon_1\partial_n \phi_1 v ds =   \int_{\Omega_1}  \sum_{k=1}^{N_q} q_k\delta(\mathbf{x},\mathbf{x}_k)  v ~d\mathbf{x}.
\end{equation}
For external problem we use BEM and the same direct formulation as in~\eqref{eq:direct}, hence we obtain
\begin{align*}
\frac{\phi_{1,\Gamma}}{2} - K_{Y}^{\Gamma}(\phi_{1,\Gamma}) + \frac{\epsilon_1}{\epsilon_2}V_{Y}^{\Gamma} \left( \frac{\partial}{\partial \mathbf{n}} \phi_{1,\Gamma} \right) & = 0.
\end{align*}

Using the above definition and assuming that $\lambda_{1,\Gamma}  = \frac{\partial}{\partial \mathbf{n}} \phi_{1,\Gamma} $ the coupling problem can be written as follows
\begin{center}
  \textit{Find $ \phi_1 \in H^1(\Omega_1)$ and $\lambda_{1,\Gamma} \in H^{-\frac{1}{2}}(\Gamma)$ such that for all $v \in H^1(\Omega_1)$ and $\zeta \in H^{-\frac{1}{2}}(\Gamma)$}
\begin{equation} 
\label{eq:standard_fem_bem}
 \left\{
 \begin{array}{rcl}
 \int_{\Omega_1} \epsilon_1 \nabla \phi_1 : \nabla v ~d\mathbf{x}  - \oint_\Gamma \epsilon_1\partial_n \phi_1 v ds &=&   \int_{\Omega_1}  \sum_{k=1}^{N_q} q_k\delta(\mathbf{x},\mathbf{x}_k)  v ~d\mathbf{x} \\[3mm] 
  \oint_\Gamma \left(\tfrac{1}{2} - K_{Y}^{\Gamma}\right) \phi_{1,\Gamma} \zeta ds + \frac{\epsilon_1}{\epsilon_2} \oint_\Gamma V_{Y}^{\Gamma} \lambda_{1,\Gamma} \zeta ds &=&0.
  \end{array}
  \right.
\end{equation}
\end{center}


%\begin{align}\label{eq:standard_fem_bem}
% \int_{\Omega_1} \epsilon_1 \nabla \phi_1 : \nabla v ~d\mathbf{x}  - \oint_\Gamma \epsilon_1\partial_n \phi_1 v ds &=   \int_{\Omega_1}  \sum_{k=1}^{N_q} q_k\delta(\mathbf{x},\mathbf{x}_k)  v ~d\mathbf{x} \\
% \oint_\Gamma \frac{\phi_{1,\Gamma}}{2} - K_{Y}^{\Gamma}(\phi_{1,\Gamma}) + \frac{\epsilon_1}{\epsilon_2}V_{Y}^{\Gamma} \left( \lambda_{1,\Gamma} \right) & = 0
%\end{align}


    \item Hybrid FEM-BEM coupling
    
  In the previous section, we focused on the standard formulation. Now, we propose the weak penalty formulation of a couple problem. The discretisation is made by using finite element methods in the domain $\Omega_1$ and boundary integral methods in the domain $\Omega_2$.
  
    The main difference between the standard and hybrid FEM-BEM formulations is addition of hybrid variable $\widetilde{\phi}$ that is a trace of $\phi_1$ and $\phi_2$ on the interface $\Gamma$ ($\widetilde{\phi} = \phi_1 = \phi_2$ on  $\Gamma$). Both FEM and BEM subproblems are modified to include the hybrid variable by adding some terms that enforce the equality of it with the trace of $\phi_1$ and $\phi_2$.
    
    The standard formulation of interior problem~\eqref{eq:fem} can be rewritten as follows
\begin{align*}
 \int_{\Omega_m} \frac{\epsilon_1(\mathbf{x})}{\epsilon_2} \nabla \phi_1 : \nabla v d\mathbf{x} -  \oint_\Gamma \frac{\epsilon_1(\mathbf{x})}{\epsilon_2}  \partial_n \phi_1 v ds
  & \\
 -  \oint_\Gamma \frac{\epsilon_1(\mathbf{x})}{\epsilon_2} \partial_n v (\phi_1 - \widetilde{\phi}) ds
  + \tfrac{\tau_F}{h} \oint_\Gamma \frac{\epsilon_1(\mathbf{x})}{\epsilon_2} (\phi_1 - \widetilde{\phi}) v ds & = \int_{\Omega_1}  \sum_{k=1}^{N_q} q_k\delta(\mathbf{x},\mathbf{x}_k)   v ~d\mathbf{x}.
\end{align*}
With every iteration we are solving above problem for given $\widetilde{\phi}$ and hence obtaining the solution $ \phi_1$.
%We are using FEniCS finite element library, hence the above formulation is more sumilar to the implementation.

In the case of exterior problem of~\eqref{eq:pbe_vp} we use Yukawa kernel and 
%matrix form, for clarity and similarity with BEMpp implementation. In this case we consider linearised formulation of the Poisson-Boltzman equation that means our operators are built as for modified Helmholtz equation with wavenumber $\kappa$.
for given $\widetilde{\phi}$, with each iteration we are solving the following system
\begin{align*}
V_{Y}^{\Gamma} \lambda_{2,\Gamma}  - \left( \tfrac{1}{2}Id + K_{Y}^{\Gamma}  \right) \phi_2 &= - \widetilde{\phi},  \\
 \left( \tfrac{1}{2} Id +  K_{Y}^{',\Gamma}\right)  \lambda_{2,\Gamma} + \left(W_{Y}^{\Gamma} + \tau_B Id\right)  \phi_2  &= \tau_B \widetilde{\phi},
\end{align*}
where $K'$ and $W$ are the adjoint double-layer potentials and hypersingular operator for the Yukawa kernel.

Both, the interior and exterior problems have unique solutions for given $\widetilde{\phi}$. In order to find $\widetilde{\phi}$ we solve the following problem on $\Gamma$
\begin{equation*}
\tfrac{\tau_F}{h} \left( \phi_1 - \widetilde{\phi}\right) + \tau_B \left( \phi_2 - \widetilde{\phi}\right)  - \left(\frac{\epsilon_1}{\epsilon_2} \frac{\partial}{\partial \mathbf{n}} \phi_1 -\lambda_{2,\Gamma} \right)  = 0.
\end{equation*}
\end{itemize}


%((Place Computational Methods here. Not needed for review articles))

%((Computational results should be prepared following the IUPAC guidelines (See Journal of Computational Chemistry, 20: 1587-1590 and 20:1591-1592). In particular, it is required that the level of theory employed is appropriate to the problem at hand, and that the sufficient details about methodology are provided to allow the work to be reproduced.)

%((In full papers, this section appears immediately after the introduction. In Rapid Communications, this section appears just before the Acknowledgments.))

\section*{\sffamily \Large RESULTS AND DISCUSSION}
This section presents the verification and performance results of the presented FEM-BEM coupling schemes, for molecules modeled as cavities with constant and varying permittivity.
With a constant permittivity inside the molecule, we tested convergence against an analytical expression of the solvation energy of a sphere \cite{Kirkwood1934}, and then compared a more realistic geometry (arginine) with a purely BEM implementation.
We also considered a Gaussian-varying permittivity\cite{grant2001smooth,li2013dielectric} inside the molecular cavity of arginine, and used APBS \cite{BakerETal2001} to verify our results.
The final tests show the scaling of the FEM-BEM coupling, as the molecule size grows. 

All runs were done on a Lenovo ThinkStation P620 with AMD Ryzen ThreadRipper PRO 3975WX (32-core and 3.5 GHz) and 128 GB RAM. 

\section*{\sffamily \Large Software environment}

For the finite element computations, we use the software package FEniCSx while for the boundary element computations, we use Bempp-cl together with Exafmm-t. FEniCSx is the successor of the widely used FEniCS finite element library.
It provides a convenient Python interface, in which problems are described using UFL (Unified Form Language), a convenient domain-specific language specifically designed for finite element discretisations of partial differential equations. During assembly, the UFL description is transformed into efficient low-level C++ code and just-in-time compiled. Bempp-cl is a Python package that uses low-level OpenCL kernels written in C99 to provide optimised assembly routines. The built-in dense assembly routines are highly efficient for moderate discretisation sizes up to a few ten thousand elements.

For very large grid sizes the user can enable fast multipole method (FMM) assembly which internally is handled in Bempp-cl through an interface to the Exafmm-t FMM library. For $N$ surface elements this reduces the memory and computational complexity from $\mathcal{O}(N^2)$ in the dense assembly case to $\mathcal{O}(N)$ in the FMM case, making large boundary element problems tractable on standard workstations.

To couple FEniCSx with Bempp-cl we load a volume mesh with FEniCSx. We then export the corresponding boundary mesh into Bempp-cl and assemble the boundary spaces there. Bempp-cl provides numerical trace operators that can translate from the degree of freedom (dof) representation in FEniCSx to the dof representation in Bempp-cl. The corresponding translation work is handled opaquely and the user only needs to deal with high-level interfaces of FEniCSx operators, Bempp-cl operators, and trace operators. FMM assembly fits automatically into this framework and can be enabled or disabled as a simple configuration option.

Docker images containing FEniCSx, Bempp-cl, and Exafmm-t are publicly available (\href{https://bempp.com/installation.html}{https://bempp.com/installation.html}), and all codes used in this sections are simple Jupyter Notebooks that can be reproducibly executed in this provided Docker image.

\section*{\sffamily \Large Results with constant permitivitty}

In implicit-solvent models, the molecule is usually considered as a region with constant permittivity, in this case, with $\epsilon_1=2$.
In the solvent region, we used the permittivity of water ($\epsilon_2$=80) and an inverse of the Debye length of $\kappa=0.125$ \AA$^{-1}$.
As in these cases there is an analytical solution for $\phi_c$ in Eq. \eqref{eq:phic}, we compute $\phi_r$ with Eq. \eqref{eq:phi_reac} in both BEM-BEM and FEM-BEM coupling approaches. Then, for FEM-BEM, the integral over $\Gamma$ in Eq. \eqref{eq:phi_reac} corresponds to the trace of the solution vector from Eq. \eqref{eq:fembem_matrix}.

\subsection*{\sffamily \large Convergence of a spherical cavity}

The Kirkwood sphere \cite{Kirkwood1934} is a standard benchmark test for the Poisson-Boltzmann equation in molecular electrostatics. 
In this case, we considered a spherical cavity of radius $R=2$ \AA, with three charges ($q_1$=1, $q_2$=1, and $q_3$=0.75) placed at $\mathbf{x}_1=(1,0,0)$, $\mathbf{x}_2=(0.7,0.7,0)$, and $\mathbf{x}_3=(-0.5,-0.5,0)$.
Figure \ref{fig:error_sphere} shows the error convergence of the FEM-BEM approaches, and a reference BEM implementation, to the analytical solution ($\Delta G_{solv}= -336.0396$ kcal/mol). 
In these runs, the FEM mesh was generated using GMSH~\cite{geuzaine2009gmsh} 
%Bempp (check?) from a surface discretization 
with %1, 2, 4, and 8 
2, 6, 21 and 83 vertices per \AA$^2$ on the SES, the surface mesh of this volume one was also used for the BEM runs. 
%For the hybrid FEM-BEM coupling approach, we used $\tau$=5.
The error in Figure \ref{fig:error_sphere} decays linearly with the surface area, which agrees with the expected convergence for P1 elements, indicating a correct implementation of the numerical scheme. 
%We can see that the purely BEM implementation outperforms (or not?) FEM-BEM coupling in terms of accuracy for equivalent meshes, and also that the hybrid approach does not (or does?) influence accuracy. 

\begin{figure}
  \centering
  \includegraphics[width=0.5\linewidth]{DolfinX_Sphere_const_coeff_error.png}
  \caption{Error for the Kirkwood sphere.}
  \label{fig:error_sphere}
\end{figure}

\subsection*{\sffamily \large Performance with arginine}

As a more realistic test, we assessed the performance of the FEM-BEM coupling technique against a purely BEM implementation for arginine.
The structure of arginine taken from the protein data bank, and parameterized with the Amber\cite{ponder2003force} force field. 
We generated surface meshes containing 4.1, 6.7, 8.6, 17, and 24.5 vertices per \AA$^2$ with Nanoshaper.~\cite{decherchi2013general}
These densities correspond to a grid scale parameter in Nanoshaper equal to 1.6, 2.0, 2.4, 3.4, and 4.0, respectively, where the grid scale is the reciprocal of the average characteristic length of the triangles.
The surface meshes were inputs to our purely BEM solver and to create the FEM mesh with pyGAMer,~\cite{lee2020open} which invoked TetGen\cite{hang2015tetgen} with a quality parameter (radius-edge ratio) of 1.0.

The solvation energy computed with the two schemes is presented by Figure \ref{fig:arg_constant_energy}, which, as expected, converges to a similar answer as the mesh is refined.
Figure \ref{fig:arg2_constant_time_iter} compares the iteration count and time-to-solution. The left plot shows that using only BEM outperforms the coupled FEM-BEM approach in terms of iterations count. However, if we look at the total time that solvers take to obtain the solution, we can see the advantage of using the FEM-BEM coupling. The higher computational cost is caused by the need of using a hypersingular operator in the BEM formulation, and the fact that we are not using any acceleration method ($ie.$ FMM). Regardless, timings for FEM-BEM coupling scale much worse with size than its pure BEM counterpart, indicating that BEM-BEM coupling is better for larger problems. 
%\textcolor{red}{Chris: I thought BEM-BEM outperformed FEM-BEM in total time. Am I wrong there?} \textcolor{blue}{Michal: Not for this formulation of BEM-BEM since we have hypersingular operator. But you can see that for denser meshes we should outperform it.} 


\begin{figure}
\centering
   \includegraphics[width=0.5\linewidth]{DolfinX_Arginine2_const_coeff_error.png}
%   \includegraphics[width=0.47\linewidth]{DolfinX_Arginine_const_coeff_error_11.png}
\caption{Solvation energy for arginine with a constant permittivity. %(left - Arginine2 and right - Arginine)
%maybe we could also add a "error" wrt extrapolation, or a plot with energy for each mesh?
}
\label{fig:arg_constant_energy}
\end{figure}

\begin{figure}
\centering
   \includegraphics[width=0.45\linewidth]{DolfinX_Arginine2_const_coeff_iter.png}
%  \includegraphics[width=0.45\linewidth]{DolfinX_Arginine2_const_coeff_time.png}
%  \includegraphics[width=0.45\linewidth]{DolfinX_Arginine2_const_coeff_setup_time.png}
  \includegraphics[width=0.45\linewidth]{DolfinX_Arginine2_const_coeff_total_time.png}
%   \includegraphics[width=0.45\linewidth]{DolfinX_Arginine_const_coeff_iter_11.png}
%%  \includegraphics[width=0.45\linewidth]{DolfinX_Arginine_const_coeff_time_11.png}
%%  \includegraphics[width=0.45\linewidth]{DolfinX_Arginine_const_coeff_setup_time_11.png}
%  \includegraphics[width=0.45\linewidth]{DolfinX_Arginine_const_coeff_total_time_11.png}
  \caption{Iteration count (left) and time-to-solution (right) for arginine with a constant permittivity. %(left: Online time taken to solve systems, right: Offline time taken to set up systems). 
%NOTE: x axis differ between top and bottom. Also, should we include preconditioned vs non preconditioned? We could easily precondition BEM-BEM too.  %(1) fix title, (2) can we put them in a single plot, (3) 
%maybe we could also add a "error" wrt extrapolation, or a plot with energy for each mesh?
}
\label{fig:arg2_constant_time_iter}
\end{figure}

%\begin{figure}
%\centering
%%   \includegraphics[width=0.45\linewidth]{DolfinX_Arginine2_const_coeff_iter.png}
%%%  \includegraphics[width=0.45\linewidth]{DolfinX_Arginine2_const_coeff_time.png}
%%%  \includegraphics[width=0.45\linewidth]{DolfinX_Arginine2_const_coeff_setup_time.png}
%%  \includegraphics[width=0.45\linewidth]{DolfinX_Arginine2_const_coeff_total_time.png}
%   \includegraphics[width=0.45\linewidth]{DolfinX_Arginine_const_coeff_iter_11.png}
%%  \includegraphics[width=0.45\linewidth]{DolfinX_Arginine_const_coeff_time_11.png}
%%  \includegraphics[width=0.45\linewidth]{DolfinX_Arginine_const_coeff_setup_time_11.png}
%  \includegraphics[width=0.45\linewidth]{DolfinX_Arginine_const_coeff_total_time_11.png}
%  \caption{Iteration count (left) and time-to-solution (right) for Arginine with a constant permittivity. %(left: Online time taken to solve systems, right: Offline time taken to set up systems). 
%%NOTE: x axis differ between top and bottom. Also, should we include preconditioned vs non preconditioned? We could easily precondition BEM-BEM too.  %(1) fix title, (2) can we put them in a single plot, (3) 
%%maybe we could also add a "error" wrt extrapolation, or a plot with energy for each mesh?
%}
%\label{fig:arg_constant_time_iter}
%\end{figure}


%\begin{figure}
%\centering
%%   \includegraphics[width=0.45\linewidth]{Arginine_const_coeff_error.png}
%  \includegraphics[width=0.45\linewidth]{Arginine_const_coeff_iter_11.png}
%%   \includegraphics[width=0.45\linewidth]{No_prec_Arginine_const_coeff_iter.png}
%%   \includegraphics[width=0.45\linewidth]{Arginine_const_coeff_time.png}
%\caption{Iteration count for arginine with a constant permittivity. %(1) fix title, (2) can we put them in a single plot, (3) hybrid is internal or external iterations? maybe we should present the total count?
%}
%\label{fig:arg_contant_iter}
%\end{figure}



\section*{\sffamily \Large Results with variable permittivity}

\subsection*{\sffamily \large Motivation: modeling the solute with a Gaussian-based variable permittivity}

In contrast to a purely BEM approach, FEM-BEM coupling gives flexibility to consider space-varying field parameters. 
A popular description of the molecule is to consider a permittivity that varies like a Gaussian around each atom,\cite{grant2001smooth} which has shown enhanced accuracy in some applications, like pKa calculations.\cite{li2013dielectric}
In this setting, we define a density function $\rho$ depending on position $r$ as
%
\begin{equation}
\rho(r) := \prod_i \left(1 - \exp{\left(\frac{\|r-\mathbf{x}_i\|}{\sigma^2 R_i^2}\right)}\right)
\end{equation}
%
where the product runs over all the atoms of the solute, $R_i$ is the van der Waals radius of atom i, and we used $\sigma$=1. Then, we can compute the permittivity as
%
\begin{equation}\label{eq:varying_eps}
\epsilon := \left(1-\rho \right) \epsilon_1 + \rho\epsilon_2
\end{equation}

As $\epsilon$ is variable, Equation \eqref{eq:phic} does not have an analytical solution, and the electrostatic potential in the vacuum state has to be computed numerically.
For vacuum calculations, we considered the same distribution of $\epsilon$ inside the molecule as in the solvated case, but the solvent permittivity was set to $\epsilon_2$=2. 
Other implementations of Gaussian permittivities also modify the solute permittivity in vacuum calculations, according to a set cutoff.\cite{li2013dielectric} We did not consider a cutoff in our calculations.



\subsection*{\sffamily \large Convergence for arginine with APBS}

We used Equation \eqref{eq:varying_eps} to generate dielectric maps, which we ran on APBS~\cite{BakerETal2001} for comparison. 
We chose APBS because it provides an easy interface to control dielectric maps, in order to ensure their agreement with the maps imposed in our FEM-BEM coupled approach.

Table \ref{table:arg_variable} shows a comparison of solvation energy computed with APBS and our FEM-BEM coupling approach. We can see that they are both converging to equivalent values, where the finest meshes agree up to 0.5\% (0.5 kcal/mol). The coarsest meshes used in both cases are within the recommended densities for accurate solvation energy simulations with constant permittivity. For example, a finite difference mesh with $h$=0.5 \AA~ or less is recommended for binding energy calculations\cite{sorensen2015comprehensive} (the result of substracting two solvation energies), and a similar study with BEM\cite{CooperBardhanBarba2014} showed that a mesh with 2 vertices per \AA$^2$ gives acceptable results when computing binding and solvation energies. We see a jump in the solvation energy for the FEM-BEM results going from 4.1 to 6.7 vertices per \AA$^2$, but later the energy monotonically decreases, converging to a solution. This is an indication that mesh requirements for variable permitivities may be tighter than with a constant permittivity, even though the jump in dielectric constant accross the molecular surface is usually smaller. 

Figure \ref{fig:arg2_variable} contains the performance results of this test case, where we distinguish the iteration count from the calculation in dissolved and vacuum states. \textcolor{red}{Chris: if we don't include Fig. \ref{fig:arg2_variable}, we should remove this paragraph. I believe that if we show performance for fem-bem, a reviewer might be tempted to ask it for APBS too, which might be difficult.}  \textcolor{blue}{Michal: I completely agree. We were even talking about not putting them, but just for our information, we kept it for now. They weren't going to be in the final version}
\begin{table}
\centering
\begin{tabular}{c|c|c|c}
&Mesh size  & Grid points & $\Delta G_{solv}$\\
&\AA       & &  kcal/mol \\
\hline
\multirow{4}{*}{APBS}& 0.52$\times$0.52$\times$0.52 & 97$\times$97$\times$97 & -107.6186 \\ 
& 0.39$\times$0.39$\times$0.39 & 129$\times$129$\times$129 & -107.8752\\ 
&0.26$\times$0.26$\times$0.26 & 193$\times$193$\times$193& -108.3378\\ 
&0.195$\times$0.195$\times$0.195 & 257$\times$257$\times$257& -108.5837\\ 
&0.098$\times$0.098$\times$0.098 & 513$\times$513$\times$513& -108.8844\\ 
\hline
&Mesh dens. & DOFs & $\Delta G_{solv}$\\
&vert/\AA$^2$   &  &  kcal/mol \\
\hline
%\multirow{5}{*}{Standard FEM-BEM}& 2 & -36.239\\
    & 4.1 & 3 491 & -109.933 \\
FEM-BEM    & 6.7  & 5 787 & -110.238 \\
coupling    & 8.57  & 8 844 & -109.658 \\
    & 17 & 19 911 & -109.368 \\
    & 24.5 & 32 302 & -109.286 \\
\hline
%\multirow{5}{*}{Standard FEM-BEM}& 2 & -29.554\\
%    & 2 & -34.670\\
%Hybrid    & 4  & -33.603 \\
%FEM-BEM    & 8  & -32.822 \\
%    & 16 & -32.072 \\
%\hline
\end{tabular}
\caption{Solvation energy of arginine with a Gaussian-like permittivity, computed using the FEM-BEM approach and APBS. The mesh density for FEM-BEM corresponds to the vertex density of the surface mesh used to generate the volumetric mesh.}
\label{table:arg_variable}
\end{table}

%\begin{figure}
%\centering
%% \includegraphics[width=0.45\linewidth]{Arginine_varying_coeff_iter.png}
%% \includegraphics[width=0.45\linewidth]{Arginine_varying_coeff_time.png}
%\includegraphics[width=0.45\linewidth]{Arginine_varying_coeff_time_11.png}
%\includegraphics[width=0.45\linewidth]{Arginine_varying_coeff_setup_time_11.png}
%\includegraphics[width=0.45\linewidth]{Arginine_varying_coeff_iter_11.png}
%\caption{Iteration count and time-to-solution to compute the solvation energy of arginine with a variable permittivity (left: Online time taken to solve systems, right: Offline time taken to set up systems), using FEM-BEM coupling. %Can we make this only two plots?
%}
%\label{fig:arg_variable}
%\end{figure}

\begin{figure}
\centering
% \includegraphics[width=0.45\linewidth]{DolfinX_Arginine2_varying_coeff_iter.png}
% \includegraphics[width=0.45\linewidth]{DolfinX_Arginine2_varying_coeff_time.png}
\includegraphics[width=0.45\linewidth]{DolfinX_Arginine2_varying_coeff_time.png}
\includegraphics[width=0.45\linewidth]{DolfinX_Arginine2_varying_coeff_set_time.png}
\includegraphics[width=0.45\linewidth]{DolfinX_Arginine2_varying_coeff_iter.png}
\caption{Iteration count and time-to-solution to compute the solvation energy of arginine with a variable permittivity (left: Online time taken to solve systems, right: Offline time taken to set up systems), using FEM-BEM coupling. %Can we make this only two plots?
\textcolor{red}{Chris: not sure if we want to show performance here. Reviewers may ask for performance with APBS too, and those were run on a different machine, and has a number of steps.}}
\label{fig:arg2_variable}
\end{figure}


\subsection*{\sffamily \large Performance analysis for larger structures}

So far, we have only tested the FEM-BEM coupled approach with small structures. Here, we study its behaviour with larger structures to evaluate its applicability in more realistic settings, and  in test cases where BEM-BEM coupling is not an alternative. Using the same Gaussian permittivity field inside the molecule, we computed the solvation free energy of protein G B1 (955 atoms, PDB code 1pgb), lysozyme (1960 atoms, PDB code 1lyz), the barnase-barstar complex (9464 atoms, PDB code 1x1u), and immunoglobulin G (20148 atoms, PDB code 1igt). All structures were parameterized with the Amber\cite{Swanson05} force field using pdb2pqr.\cite{Dolinsky04} The surface was meshed with Nanoshaper\cite{decherchi2013general} considering a grid scale of 1.5, and then used to mesh the solute region with pyGAMer\cite{lee2020open} and TetGen\cite{hang2015tetgen} with a radius-edge ratio of 1.0. Solvation free energy and timings for these runs are presented in Table \ref{table:large_variable}, showing that our FEM-BEM coupling method can reach medium-to-large sized proteins on a workstation. For these larger structures we enabled the FMM capabilities of Bempp-cl.

Table \ref{table:large_variable} shows that the iteration count increases with the problem size. This is expected for Johnson-N\'ed\'elec, which is the simplest coupling strategy. To isolate the effect of the increased iterations in the analysis, we separate timings in ``setup'' and ``solving'' time, where the ``setup'' time is independent of the number of iterations, and the ``solving'' time corresponds to the time spent in the GMRES solver. Then, the time-per-iteration in Table \ref{table:large_variable} only considers the ``solving'' time. Looking at the increase in degrees of freedom (DOFs), which grows equivalently for FEM and BEM, the ``setup'' time scales slightly worse than $\mathcal{O}(N^2)$, but represents the smalle fraction of the total time. On the other hand, the ``solving time per iteration'' scales closer to $\mathcal{O}(N)$, which is expected, as we are using a FMM for the BEM portion of the matrix. This is an indication that the high ``solving time'' is mainly due to the increase in iteration count, and having better conditioned coupling methods, such as the so-called hybrid approach,\cite{betcke2022hybrid} can have a large impact in the time to solution.

Even though this scheme is capable of calculating the electrostatic potential in medium-to-large proteins, the largest test case in Table \ref{table:large_variable} (1igt) used up half of the available RAM memory \textcolor{red}{check}. 
%(\textcolor{red}{check and rewrite \textcolor{blue}{we are using around 50\% of memory}}). 
If we were aiming at larger structures, such as full viruses,\cite{MartinezETal2019,wang2021high} we would require to improve not only in the conditioning of the system, but also in the optimization of our fast algorithms\cite{wang2021exafmm,kailasa2023pyexafmm} and parallelization.

\begin{table}
\centering
\resizebox{\textwidth}{!}{\begin{tabular}{c|c|c|c|c|c|c|c|c}
Molecule & FEM  & BEM &  $\Delta G_{solv}$ & Iterations & Setup Time & Solving Time & Solv. time per & Total Time \\
& DOFs & DOFs &  kcal/mol & & [sec] & [sec] & iter. [sec] & [sec] \\
\hline
 1pgb  & 29 434 &  10 058 &  -300.888 & 826 & 82 & 547 & 0.6 & 6 629  \\
 1lyz  & 56 114 & 18 606 &  -599.642 & 1 285 & 330 & 1 640  & 1.2 & 1 970  \\
 %1lyz gs3 & 257 391 & 75 048 &  -600.637 & 2 949  & 1 690 & 10 900 & 3.6 & 12 590   \\
 1x1u & 263 181 & 81 258 \textcolor{red}{82360} \textcolor{blue}{???}&  -1 984.468 & 4 167 & 8 240 & 16 300 & 3.91 & 24 540  \\
 1igt & 597 575 & 187 712 \textcolor{red}{189960}  \textcolor{blue}{???}&   -3 294.230 & 8 361 & 41 200 & 64 200 & 7.67  &  105 400   \\
 \hline
\end{tabular}}
\caption{Results for larger structures with a variable permitivitty.}
\label{table:large_variable}
\end{table}
%
%\begin{table}
%\centering
%\begin{tabular}{c|c|c|c|c|c|c|c|c}
%Molecule & Mesh size  & FEM  & BEM &  $\Delta G_{solv}$ & Iterations & Setup Time & Solving Time & Total Time \\
%& vert/\AA$^2$ & DOFs & DOFs &  kcal/mol & & sec & sec & sec \\
%\hline
% 1pgb &  & 29 434 &  10 058 &  -300.888 & 826 & 82 & 547 & 629  \\
% 1lyz &  & 56 114 & 18 606 &  -599.642 & 1 285 & 330 & 1 640  & 1 970  \\
% 1lyz gs3 &  & 257 391 & 75 048 &  -600.637 & 2 949  & 1 690 & 10 900 & 12 590   \\
% 1uxu &  & 263 181 & 81 258 &  -1 984.468 & 4 167 & 8 240 & 16 300 & 24 540  \\
% 1igt &  & 597 575 & 187 712 &   -3 294.230 & 8 361 & 41 200 & 64 200 &  105 400   \\
% \hline
%\end{tabular}
%\caption{Large tests' results}
%\label{table:large_variable}
%\end{table}

%((Place Results here. Not needed for review articles.))

%\section*{\sffamily \Large First-order heading}
 
%((Equations should be inserted using standard LaTeX equation and eqnarray environments, not as graphics, and should be set in the main text))
%Equation											(1)
%((References should be superscripted and appear after punctuation.1,2 Please define all acronyms at their first usage except IR, UV, NMR, and DNA or similar commonly understood terms.)) 


%\subsection*{\sffamily \large Second-order heading}

%\subsubsection*{\sffamily \normalsize Third-order heading}

%{\sffamily \small Fourth-order heading}\\

%\section*{\sffamily \Large DISCUSSION}

%((Place Discussion here. Not needed for review articles.))


\section*{\sffamily \Large CONCLUSIONS}
This paper presents the first implementation of a FEM-BEM coupling approach to solve the Poisson--Boltzmann equation for molecular electrostatics. This brings the best of both worlds: the accuracy and efficiency of BEM to exactly enforce the boundary conditions at infinity, and the flexibility of FEM to account for space variations of the material properties and nonlinearities. After presenting verification results for a sphere and arginine with a constant permittivity inside the solute, we showcased our implementation with an advanced modeling technique that considers Gaussian-varying permittivities in a confined region, with the results validated against the widely-used APBS software. The final scaling results for larger molecules start from protein G B1 (955 atoms) and go up to immunoglobulin G (20\,148 atoms), proving the applicability of this approach for realistic problems. Even though our implementation was able to reach medium-to-large systems, we recongnize the need for further research towards better preconditioning of the linear system, optimizing the coupling technique, acceleration algorithms, and parallel execution, especially as we look towards much larger solutes, like viruses. We hope this proof-of-concept work will serve as motivation for future model development that considers space-varying permittivities and Debye lengths, and mixed linear-nonlinear techniques, especially for highly-charged systems, like nucleic acids. 

%((Place Conclusions here.))



\subsection*{\sffamily \large ACKNOWLEDGMENTS}
MB acknowledges the support from Kingston University through First Kingston University Grant. \\
CDC acknowledges the support from CCTVal through ANID PIA/APOYO AFB220004.\\
MS needs to look up the grant he has to put here. \\
We would like to thank J{\o}rgen Dokken for his help with various implementation details related to FEniCSx.
%((Place Acknowledgments here))

\subsection*{\sffamily \large DATA AVAILABILITY STATEMENT}
All codes to reproduce the results of this manuscript can be found in the Github repository \url{https://github.com/MichalBosy/FEM_BEM_coupling/}, alongside links to Docker images
that include the codes alongside installations of appropriate versions of Bempp-cl and FEniCSx.

%((Additional Supporting Information may be found in the online version of this article.))

\clearpage

%%%%%%%%%%%%%%%%%%%%%%%%%%%%%%%%%%%%%%%%%%%%%%%%%%%%%%%%%%%%%%%%%%%%%%%%%%%%%%%%%
% BIBLIOGRAPHY

\bibliography{main}   % Produces the bibliography via BibTeX.

%\begin{thebibliography}{99}
%
%
%\bibitem{Coulson}
%Coulson, C. A., Rev. Mod. Phys., \textbf{1960}, 32,170-177.
%\bibitem{Malrieu}
%Malrieu, J.-P., J. Mol. Struct., \textbf{1998}, 424, 1-2,83-91.
%\bibitem{Shaik}
%Shaik, S., New. J. Chem., \textbf{2007}, 31,2015-2028.
%\bibitem{Hoffmann}
%Hoffman, R., Schleyer, P. v. R., Schaefer III, H. F., \textbf{2008}, 47, 7164-7167.
%\bibitem{Perdew}
%Perdew, J. P., Ruzsinszky, A., Constantin, L., Sun, J., Csonka, G., J. Chem. Theory Comput., \textbf{2009}, 5, 902-908.
%\bibitem{Koros}
%Koros, W. J.; Chern, R. T. In Handbook of Separation Process Technology; Rousseau, E. D.; Russell, B., Eds.; Wiley: New York, \textbf{1987}; Vol. 2, Chapter 20, pp 34-45.
%\end{thebibliography}


%%%%%%%%%%%%%%%%%%%%%%%%%%%%%%%%%%%%%%%%%%%%%%%%%%%%%%%%%%%%%%%%%%%%%%%%%%%%%%%%%

%\clearpage
%%%%%%%%%%%%%%%%%%%%%%%%%%%%%%%%%%%%%%%%%%%%%%%%%%%%%%%%%%%%%%%%%%%%%%%%%%%%%%%%%%
%% FIGURE CAPTIONS
%
%%%%%% FIGURE ---- cc.eps
%\begin{figure}
%\caption{\label{cc} Place Figure 1 caption here. In the case of reproduced figures in review articles, you must obtain the publisher's permission and state a suitable notice here along with a citation.}
%\end{figure}
%
%\begin{figure}
%\caption{\label{fig2} Place Figure 2 caption here. Figures should be uploaded as individual files, preferably .tif or .eps files, at high enough resolution (600 to 1200 dpi) to ensure clarity. Please see the author’s guide for more details and specifications. For high quality illustrations, we highly recommend the use of the TikZ package.}
%\end{figure}
%
%
%%%%%%%%%%%%%%%%%%%%%%%%%%%%%
%
%
%
%%%%%%%%%%%%%%%%%%%%%%%%%%%%%%%%%%%%%%%%%%%%%%%%%%%%%%%%%%%%%%%%%%%%%%%%%%%%%%%%%%
%% FIGURE FILES
%
%\clearpage
%
%%\vspace*{0.1in}   %%% FIGURE 1
%%\begin{center}
%%\includegraphics[width=0.2\columnwidth,keepaspectratio=true]{cc.eps}
%%\end{center}
%\vspace{0.25in}
%\hspace*{3in}
%{\Large
%\begin{minipage}[t]{3in}
%\baselineskip = .5\baselineskip
%Figure 1 \\
%Author A, Author B, Author C, Author D \\
%J.\ Comput.\ Chem.
%\end{minipage}
%}
%
%\clearpage
%
%\begin{table}
%\begin{tabular}{|c|c|c|c|}\hline
%\textbf{Quantity} & \textbf{Calculated} & \textbf{Observed} & \textbf{Error} \\ \hline
%  Density & 5.3 & 6.3 & Within limits \\ \hline
%  Optical magnification & 8.3 & 90.9 & Utterly unacceptable\! \\ \hline
%\end{tabular}
%\caption{\label{tbl1} Place table caption here.}
%\end{table}

\end{document}

