This paper presents the first implementation of a FEM-BEM coupling approach to solve the Poisson--Boltzmann equation for molecular electrostatics. This brings the best of both worlds: the accuracy and efficiency of BEM to exactly enforce the boundary conditions at infinity, and the flexibility of FEM to account for space variations of the material properties and nonlinearities. After presenting verification results of solvation energy for a sphere and arginine with a constant permittivity inside the solute, and binding energy for an extensive data set, we showcased our implementation with an advanced modelling technique that considers Gaussian-varying permittivities in a confined region, with the results validated against the widely-used APBS software. The final scaling results for larger molecules start from protein G B1 (955 atoms) and go up to immunoglobulin G (20\,148 atoms), proving the applicability of this approach for realistic problems. Even though our implementation was able to reach medium-to-large systems, we recognise the need for further research towards better preconditioning of the linear system, optimising the coupling technique, acceleration algorithms, and parallel execution, especially as we look towards much larger solutes, like viruses. We hope this proof-of-concept work will serve as motivation for future model development that considers space-varying permittivities and Debye lengths, and mixed linear-nonlinear techniques, especially for highly-charged systems, like nucleic acids. 
